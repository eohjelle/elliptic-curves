\section{Lecture 2 (Wednesday 1/13)}
Last time, we outlined the basic material. Today, we start proving things.

There are three characterizations of what an elliptic curve is. Let \( k \) be a field. When convenient we will make the simplifying assumption that the characteristic of \( k \) is 2 or 3.
\begin{proposition}
  There are three equivalent ways of defining an elliptic curve:
  \begin{enumerate}[(A)]
  \item As a smooth projective genus 1 curve \( E \), with a rational point \( O \in E(k) \).
  \item As plane cubic given by the equation \[ Y^2 Z + a_1 XYZ + a_3 YZ^2 = X^3 + a_2 X^2 Z + a_4 XZ^2 + a_6 Z^3 \]
    in \( \mathbb{P}^2 \), where \( a_i \in k \) and discriminant \( \Delta \left( a_1,a_2,a_3,a_4,a_5,a_6 \right) \neq 0 \).
    (The point \( O \) corresponds to \( [0:1:0] \).)
  \item \( (E,O) \) is a proper group variety of dimension 1 with identity \( O \).
    (Recall that a group variety is a variety with group structure such that multiplication and inversion are algebraic maps, i. e. a group object in a certain category.)
  \end{enumerate}
\end{proposition}
Last time we started with (B), which is the most concrete.
Today we will establish the equivalence of (A) and (B), which is the study of projective geometry of genus 1 curves.
The equivalence of (A), (B), and (C) will be established next time.
%To show that (1) is equivalent to (3) we must prove two things, namely that a smooth projective genus 1 curve with a rational point has a group structure, and the only proper group variety in dimension 1 must be of genus 1.

The general way to study a smooth projective curve is to map it into projective space, and such a map is given by a line bundle with a generating set of global sections.

\subsection{Reminder on projective geometry of curves}
If \( C \) is a smooth projective curve, \( v \in C \) a closed points, we can measure the order of functions on \( C \) along \( v \) using that \( \mathcal{O}_{C,v} \) is a DVR.
We have
\[ \operatorname{Prin}(C) = \left\lbrace \operatorname{div} f , f \in K(C) \right\rbrace \subset \operatorname{Div}(C) = \bigoplus_{v \text{ closed point}} \mathbb{Z}v,  \]
where \( K(C) \) is the field of rational functions on \( C \), i. e. functions that are defined away from finitely many points, and \( \operatorname{div}(f) = \sum_{v \in C} \operatorname{ord}_vf v \).
The class group of \( C \) is \( \operatorname{Cl}(C) = \operatorname{Div}(C) / \operatorname{Prin}(C) \), and there is a degree map
\( \operatorname{deg} \colon \operatorname{Cl}(C) \to \mathbb{Z} \) given by \( \sum n_v v \mapsto \sum n_v  \).
In our situation there is an isomorphism \( \operatorname{Cl}(C) \cong \operatorname{Pic}(C) \), where \( \operatorname{Pic}(C) \) is the Picard group of isomorphism classes of line bundles on \( C \).
Via this isomorphism, a divisor \( D \in \operatorname{Cl}(C) \) corresponds to the line bundle \( \mathcal{O}(D) \) given by
\[ \mathcal{O}(D) (U) = \left\lbrace f \in K(C) : \operatorname{div}(f) + D \geq 0 \right\rbrace \cup \left\lbrace 0 \right\rbrace . \]

For doing projective geometry the Picard group \( \operatorname{Pic}(C) \) is important because it ``controls'' maps of \( C \) into projective space.
The reason is that if we have a set of elements \( s_0,s_1,\dots,s_d \in H^0 \left( C,\mathcal{L} \right) \) that generate \( \mathcal{L} \), we get a map
\begin{align*}
  C & \to \mathbb{P}^d \\
  v & \mapsto \left[ s_0(v) : \cdots : s_d(v) \right] .
\end{align*}
Although the value \( s_i(v) \) is not well-defined, the common ratio of the \( s_i(v) \) is well-defined as an element of \( \mathbb{P}^d \).
The way this could go wrong is if all the \( s_i \) vanish at a point, but the assumption that they generate \( \mathcal{L} \) is precisely the condition that this doesn't happen.
\begin{remark}
  More intrinsically, we get a map to projective space by sending \( v \in C \) to the hyperplane of sections in \( H^0 \left( C,\mathcal{L} \right) \) vanishing at \( v \). Depending on conventions, this is a map \( C \to \mathbb{P} \left( H^0 \left( C, \mathcal{L} \right) \right) \) or \( C \to \mathbb{P} \left( H^0 \left( C, \mathcal{L} \right)^\vee \right) \).
\end{remark}

Because of the relation between line bundles and maps to projective space, the dimension of \( H^0 \left( C, \mathcal{L} \right) \) gives an upper bound for the dimension \( d \) of projective space that we can map into.
Riemann-Roch provides the main tool for computing the dimension of global sections of line bundles.

If \( \mathcal{F} \) is a sheaf on \( C \), then the Euler characteristic is
\[ \chi(\mathcal{F}) = \operatorname{dim} H^0 \left( C,\mathcal{F} \right) - \operatorname{dim} H^1 \left( C, \mathcal{F} \right) . \]
The cohomology groups \( H^i \left( C, \mathcal{F} \right) \) are finite dimensional over \( k \) because \( C \) is projective, and vanish for \( i > 1 \) because \( C \) is a curve.

The Euler characteristic \( \chi \) is additive on short exact sequences of sheaves.
For example, if \( v \in C \) is a closed point there is an exact sequence \( 0 \to \mathcal{O}_C \to \mathcal{O}(v) \to k_v \to 0 \), where \( k_v \) is the skyscraper sheaf with value \( k \) supported at \( v \), so \( \chi(\mathcal{O}(v)) = \chi( \mathcal{O}_C ) + \operatorname{deg}v \).
More generally, \begin{equation} \label{eq:riemann-roch-1} \chi(\mathcal{O}(D)) = \chi(\mathcal{O}_C) + \operatorname{deg} D . \end{equation}

Whereas additivity of Euler characteristics is purely formal, \textit{Serre duality} is a fundamental fact which says that
\begin{equation} \label{eq:serre-duality} H^1 \left( C,\mathcal{L} \right) \cong H^0 \left( C, \Omega^1 \otimes \mathcal{L}^{-1} \right)^\vee,  \end{equation}
where \( \Omega^1 \) is the sheaf of differentials.

Combining Equations (\ref{eq:riemann-roch-1}) and (\ref{eq:serre-duality}),
\begin{align*}
  \label{eq:riemann-roch-2}
  \operatorname{dim}_k H^0 \left( C, \mathcal{O}(D) \right) - \operatorname{dim}_k H^0 \left( C, \Omega^1 \otimes \mathcal{O}(-D) \right)
  =  \chi(\mathcal{O}(D))
  = \chi(\mathcal{O}_C) + \operatorname{deg} D
  = 1 - g + \operatorname{deg} D , && \mbox{(Riemann-Roch)}
\end{align*}
where \( g = \operatorname{dim} H^1 \left( C,\mathcal{O}_C \right) = \operatorname{dim} H^0 \left( C, \Omega^1 \right) \) is the \textit{genus} of \( C \).
\begin{remark}
If \( C \) is a proper curve, but not necessarily smooth, then everything above works except for Serre duality.
The number \( \operatorname{dim} H^1 \left( C,\mathcal{O}_C \right) \) is the \textit{arithmetic genus} of \( C \).
\end{remark}
In order to apply Riemann-Roch, we will use:

\textbf{Easy fact.}  If \( \operatorname{deg} \mathcal{L} \leq 0 \) and \( \mathcal{L} \not\cong \mathcal{O}_C \), then \( \operatorname{dim} H^0 \left( C,\mathcal{L} \right) = 0 \).

\subsection{Back to elliptic curves}
Let \( E \) be a genus 1 smooth projective curve, and \( O \in E(k) \).
The idea of going from (A) to (B) is to study maps from \( E \) into projective space determined by sections of the line bundles \( \mathcal{O}(nO) \).
Being genus 1 implies that
\begin{enumerate}[(i)]
\item Riemann-Roch becomes \( \operatorname{dim}_k H^0 \left( E, \mathcal{O}(nO) \right) - \operatorname{dim}_k H^0 \left( E, \Omega^1 \otimes \mathcal{O}(-nO)  \right) = n .  \)
\item We have \( \Omega^1 \cong \mathcal{O}_C \).  This is because first of all, \( \operatorname{dim} H^0 \left( E, \Omega^1 \right) = g = 1 \), and second, because by Riemann-Roch \( \operatorname{deg} \Omega^1 = 2g - 2 = 0 \). Hence the statement follows from the ``easy fact''.
\end{enumerate}
Combining (i) and (ii), we obtain
\( \operatorname{dim}_k H^0 \left( E, \mathcal{O}(nO) \right) - \operatorname{dim}_k H^0 \left( E, \mathcal{O}(-nO) \right) = n . \)
Using the ``easy fact'' that \( H^0 \left( E, \mathcal{O}(D) \right) = 0 \) if \( \operatorname{deg} D < 0 \), it follows that
\[ \operatorname{dim}_k H^0 \left( E, \mathcal{O}(nO) \right) =
  \begin{cases}
    1, & n = 0, \\
    n, & n > 0 .
  \end{cases}
\]
Note that \( H^0 \left( E, \mathcal{O}(0 O) \right) \subseteq H^0 \left( E, \mathcal{O}(O) \right) \subseteq H^0 \left( E, \mathcal{O}(2O) \right) \subseteq \cdots \).
The constant function \( 1 \in H^0 \left( E, \mathcal{O}(0) \right) \) is a non-zero element.
Since \( H^0 \left( E, \mathcal{O}(2O) \right) \) has dimension 2 there is an element \( x \in H^0 \left( E, \mathcal{O}(2O) \right) \) that is not a scalar multiple of 1.
And since \( H^0 \left( E, \mathcal{O}(3O) \right)  \) has dimension 3 there is an element \( y \in H^0 \left( E, \mathcal{O}(3O) \right) \) that is not a linear combination of 1 and \( x \).

\begin{claim}
  \label{res:E-is-cubic-claim}
  The map
\begin{align*}
  E & \to \mathbb{P}^2 \\
  v & \mapsto \left[ x(v) : y(v) : 1 \right]
\end{align*}
is an embedding, and the image is a smooth plane cubic given by a Weierstrass equation.
\end{claim}

There are two things to check, namely that the map is an \textit{embedding} and that the image is a smooth plane cubic.

To guess the image, we find relations between monomials in \( x,y,1 \).
In this case, the target projective space is 2-dimensional, so to identify the image it will suffice to find one relation, as this cuts out a 1-dimensional object.

Note that \( x^3,x^2,x,y,y^2,xy,1 \in H^0 \left( E, \mathcal{O}(6O) \right) \).
So we have 7 elements of a 6-dimensional space, so there exists a non-trivial linear relation between them.
Because pole orders are distinct except for \( x^3, y^2 \), this relation has to involve \( x^3 \) and \( y^2 \).
After rescaling \( x,y \), we can make this relation of the form
\[ y^2 + a_1 xy + a_3 y = x^3 + a_2 x^2 + a_4 x + a_6 . \]
\begin{remark}
  This linear relation is essentially unique, because if we throw out \( x^3 \) or \( y^2 \) the remaining elements constitute a basis.
\end{remark}
Let \( \widetilde{E} \) denote the plane cubic described by this Weierstrass equation.
In order to establish Claim \ref{res:E-is-cubic-claim}, it remains to show that
\begin{enumerate}[(i)]
\item \( \widetilde{E} \) is smooth,
\item The map \( \pi \colon E \to \widetilde{E} \) is an isomorphism.
\end{enumerate}
To do this, we have to do one basic computation. The idea is to first show that \( \pi \) is birational, and then compute the arithmetic genus.
\begin{claim}
  \( \pi \) is birational, i. e. induces an isomorphism of function fields.
\end{claim}
\textit{Proof.}
To prove this claim, we compute the (scheme-theoretic) fiber of a point \( [0:1:0] \in \widetilde{E} \).
An affine chart around \( [0:1:0] \) is \( \left[ \frac{X}{Y} : 1 : \frac{Z}{Y} \right] \), so \( \widetilde{E} \) in this chart is given by
\[ \frac{Z}{Y} + a_1 \frac{X}{Y} \frac{Z}{Y} + a_3 \left( \frac{Z}{Y} \right)^2 = \left( \frac{X}{Y} \right)^3 + a_2 \left( \frac{X}{Y} \right)^2 \frac{Z}{Y} + a_4 \frac{X}{Y} \left( \frac{Z}{Y} \right)^2 + a_6 \left( \frac{Z}{Y} \right)^3 . \]
The term \( \frac{Z}{Y} \) is linear whereas the other terms are quadratic or cubic, and this implies that the point \( [0:1:0] \) is smooth.
The inverse image \( \pi^{-1} \left\lbrace [0:1:0] \right\rbrace  \) is the locus in \( E \) where \( \frac{x}{y} \) and \( \frac{1}{y} \) has zeros.
By construction, \( y \) only has poles at \( O \), and therefore \( \pi^{-1} \left\lbrace [0:1:0] \right\rbrace = \lbrace O \rbrace \).
There is no multiplicity because \( \operatorname{ord}_O \frac{x}{y} = \operatorname{ord}_O x - \operatorname{ord}_O y = 1 \).
This shows that \( \pi \) is birational.
\qed

There are now two ways to proceed.
Silverman's approach is to show that a singular plane cubic is birational to \( \mathbb{P}^1 \), which is a contradiction because \( E \) is a genus 1 curve which is not birational to \( \mathbb{P}^1 \).
The map \( \pi \) is finite, so we have a short exact sequence of structure sheaves
\[
0 \to \mathcal{O}_{\widetilde{E}} \to \pi_* \mathcal{O}_E \to \text{skyscrapers} \to 0 ,
\]
the cokernel being a sum of skyscraper sheaves since \( \pi \) is birational (and the first map is an isomorphism over an open set).
Since \( \pi \) is finite, cohomology commutes with \( \pi_* \), so in particular \( \chi( \pi_* \mathcal{O}_E ) = \chi( \mathcal{O}_E ) = 1-1 = 0 \).
Additivity of Euler characteristics applied to the short exact sequence yields
\[ \chi( \mathcal{O}_{\widetilde{E}} ) + \chi( \text{skyscrapers} ) =  \chi( \pi_* \mathcal{O}_E ) = 0 . \]
On the other hand, there is a short exact sequence
\[ 0 \to \mathcal{O}_{\mathbb{P}^2} \left( -3 \right) \to \mathcal{O}_{\mathbb{P}^2} \to i_* \mathcal{O}_{\widetilde{E}} \to 0 , \]
where the first map is multiplication by the cubic defining \( \widetilde{E} \), and \( i \colon \widetilde{E} \hookrightarrow \mathbb{P}^2 \) is the inclusion.
Again, additivity of Euler characteristics imply that
\[ \chi( \mathcal{O}_{\mathbb{P}^2}(-3) ) + \chi( \mathcal{O}_{\widetilde{E}}) = \chi( \mathcal{O}_{\mathbb{P}^2} ) , \]
from which we can compute that \( \chi( \mathcal{O}_{\widetilde{E}} ) = 0 \).
\begin{exercise}
  Generalize this last argument to show that if \( C \) is a plane curve of degree \( d \) in \( \mathbb{P}^2 \), then \( \chi( \mathcal{O}_C ) = 1 - \frac{(d-1)(d-2)}{2} \).
\end{exercise}
The conclusion is that \( H^0 \left( E, \text{skyscrapers} \right) = \chi( \text{skyscrapers} ) = 0 \), so in fact the map
\[ \mathcal{O}_{\widetilde{E}} \to \pi_* \mathcal{O}_E \]
must be an isomorphism.
This concludes our proof of ``(A) implies (B)''.
But the computation of \( \chi( \mathcal{O}_{\widetilde{E}}) \) also shows that ``(B) implies (A)''.

To sum up, we started with \( (E,O) \) as in (1) and produced a cubic curve given by a Weierstrass equation.
But \( (E,O) \) can produce many such Weierstrass equations depending of choices of \( x,y \).
The choice is only unique up to an action of an upper triangular \( 3\times 3 \)-matrix.
This is because we chose the basis in such a way as to be compatible with the filtration
\[ H^0 \left( E, \mathcal{O}_E \right) \subset H^0 \left( E, \mathcal{O}(2O) \right) \subset H^0 \left( E, \mathcal{O}(3O) \right),  \]
i. e. we can replace \( x \mapsto u x+\alpha \) and \( y \mapsto vy + \beta x + \gamma \), and this is encoded by an upper triangular matrix.
In the end, we have
\[ \left\lbrace (E,O) \right\rbrace/\text{iso} \cong \left( k^5 \right)^{\Delta \neq 0} / B(k) . \]
If \( \operatorname{char}k \neq 2,3 \), we can further simplify this quotient, by choosing \( x,y \) such that \( a_1,a_3,a_2=0 \).
In the Weierstrass equation, we can complete the square on the left hand side and complete the cube on the right hand side:
\begin{align*}
  y^2 + a_1 xy + a_3 y & = \left( y+ \frac{a_1}{2} x + \frac{a_3}{2} \right)^2 + \text{quadratic polynomial in \( x \)} , \\
  x^3 + a_2 x^2 + a_4 x + a_6 &= \left( x + \frac{a_2}{3} \right)^3 + \text{linear polynomial in \( x \)} .
\end{align*}
This gets rid of \( a_1,a_3,a_2 \).
So if \( \operatorname{char} k \neq 2,3 \),
\begin{align*}
  \left\lbrace (E,O) \right\rbrace/\text{iso} \cong \left( k^2 \right)^{\Delta \neq 0} / k^\times = \left\lbrace \left( a_4,a_6 \right) : 4 a_4^3 + 27a_6^2 \neq 0 \right\rbrace / k^\times ,
\end{align*}
where as we'll see next time the action of \( u \in k^\times \) is given by \( u \left( a_4,a_6 \right) = \left( u^{-4} a_4, u^{-6} a_6 \right) \).


%%% Local Variables:
%%% mode: latex
%%% TeX-master: "elliptic-curves"
%%% End:
