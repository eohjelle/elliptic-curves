\section{Lecture 6 (Monday 1/25)}
Last time we gave various characterizations for an isogeny to be separable.
The most convenient one was that an isogeny \( \phi \colon E \to E' \) is separable if and only if \( (d\phi)_O \colon T_O E \to T_{O'}E' \) is non-zero (or equivalently an isomorphism because we are dimension 1).
This is equivalent to \( \phi \colon E \to E' \) being an unramified Galois cover (with Galois group \( \ker \phi ( \overline{k} ) \)).

In general, \( \phi \colon E \to E' \) identifies \( E' \) as the quotient \( E / \ker \phi \).
In the diagram
\[
\begin{tikzcd}
  E \arrow[rr,"\phi"] \arrow[dr,"f"] & & E' \arrow[dl,dashed] \\
  & C & 
\end{tikzcd}
\]
the dashed arrow exists if and only if \( f \) is invariant under translation by elements of \( \ker \phi  \).

We showed that \( \operatorname{End}(E) = \left\lbrace \text{isogenies } \phi \colon E \to E \right\rbrace \) is a characteristic zero domain, and \( [n] \neq 0 \).
To analyze \( [n] \) further we studied the effect of \( \phi^* \colon \operatorname{Pic}(E') \to \operatorname{Pic}(E) \).
There are two basic cases:
\begin{enumerate}
\item If \( \mathcal{L} \) is symmetric, \( [n]^* \mathcal{L} \cong \mathcal{L}^{\otimes n^2} \).
\item If \( \mathcal{L} \) is anti-symmetric, \( \left( \phi + \psi \right)^* \mathcal{L} \cong \phi^* \mathcal{L} \otimes \psi^* \mathcal{L} \).  
\end{enumerate}
Both facts are proved using the Theorem of the Square: If \( \phi,\psi \colon E \to E' \), then pulling back behaves like a quadratic function, and we have a ``parallelogram law''
\[ (\phi + \psi)^* \mathcal{L} + (\phi - \psi)^* \mathcal{L} \cong (\phi^* \mathcal{L})^{\otimes 2} \otimes \psi^* \mathcal{L} \otimes (-\psi)^* \mathcal{L} . \]
\textit{Proof of the Theorem of the Square.}
The first step is reduction to a ``universal case''.
Consider the 4 maps \( p_1,p_2,d,m \colon E \times E \to E \) given by projections \( p_i \), the difference \( d(P,Q) = P \ominus Q \) and addition \( m(P,Q) = P\oplus Q \).
\begin{claim}
  Let \( \mathcal{L} \in \operatorname{Pic}(E) \). Then \[ m^* \mathcal{L} \otimes d^* \mathcal{L} \cong (p_1^* \mathcal{L})^{\otimes 2} \otimes p_2^* \mathcal{L} \otimes (-p_2)^* \mathcal{L} . \]
\end{claim}
This implies the Theorem of the Square, by pulling back the isomorphism via \( (\phi,\psi) \colon E \to E'\times E' \). So it remains to prove the claim.

Let \[ \widetilde{\mathcal{L}} = m^* \mathcal{L} \otimes d^* \mathcal{L} (p_1^* \mathcal{L})^{\otimes (-2)} \otimes (p_1^* \mathcal{L})^{\otimes (-1)} \otimes ((-p_1)^*\mathcal{L})^{\otimes (-1)} . \]
We must prove that \( \widetilde{\mathcal{L}} \) is trivial.
Given a point \( Q \) in \( E \), note that
\[ \widetilde{\mathcal{L}}|_{E \times \left\lbrace Q \right\rbrace} \cong t_Q^* \mathcal{L} \otimes t_{-Q}^* \mathcal{L} \otimes \mathcal{L}^{\otimes (-2)} , \]
where \( t_{\pm Q} \colon E \to E \) is translation by \( \pm Q \).
A necessary condition of \( \widetilde{\mathcal{L}} \) to be trivial is that \( \widetilde{\mathcal{L}} |_{E \times \left\lbrace Q \right\rbrace} \) is trivial.
And this is indeed holds: It suffices to cheack for \( \mathcal{L} = \mathcal{O}(P) \), and
\begin{align*}
  t_Q^* \mathcal{O}(P) \otimes t_{-Q}^* \mathcal{O}(P) & \cong \mathcal{O} (P \oplus Q) \otimes \mathcal{O}(P \ominus Q)  \\
                                                       & \cong \mathcal{O}(P \oplus Q - O) \otimes \mathcal{O}(P \ominus Q - O) \otimes \mathcal{O}(O)^{\otimes 2} \\
                                                       & \cong \mathcal{O}(P - O) \otimes \mathcal{O}(Q-O) \otimes \mathcal{O}(P-O) \otimes \mathcal{O}(Q-O)^{\otimes (-1)} \otimes \mathcal{O}(O)^{\otimes 2} \\
                                                       & \cong \mathcal{O}(P)^{\otimes 2} ,
\end{align*}
as needed.
In fact, we checked that for any map \( i \colon \operatorname{Spec} K \to E \), where \( K \) is a field, \( (1 \times i)^* \widetilde{\mathcal{L}} \) is trivial on \( E \times \operatorname{Spec} K \).
In particular, for \( i \colon \operatorname{Spec} K(E) \to E \) the generic point of \( E \).
\todo[inline]{Insert picture}
So there is an isomorphism \( \mathcal{O}_{E \times \operatorname{Spec} K(E)} \cong \widetilde{\mathcal{L}}|_{E \times \operatorname{Spec} K(E)}  \).
It follows that there exists an open dense \( U \subseteq E \) such that \( \mathcal{O}_{E \times U} \cong \widetilde{\mathcal{L}}_{E \times U} \).
In other words, \( \widetilde{\mathcal{L}} \) is trivial away from finitely many points \( E \times \left\lbrace P_i \right\rbrace \).
Therefore, the divisor class of \( \widetilde{\mathcal{L}} \) is of the class of \( \sum_i n_i E \times \left\lbrace P_i \right\rbrace \), and it follows that
\( \widetilde{\mathcal{L}} \cong p_2^* \mathcal{O} \left( \sum_i n_i P_i \right) \).

We have shown that \( \widetilde{\mathcal{L}} \cong p_2^* \mathcal{N} \) for a line bundle \( \mathcal{N} \in \operatorname{Pic}(E) \). But then
\[ \mathcal{N} \cong \widetilde{\mathcal{L}}|_{\left\lbrace O \right\rbrace \times E} \cong \mathcal{L} \otimes [-1]^* \mathcal{L} \otimes \mathcal{L}^{\otimes (-1)} \otimes [-1]^* \mathcal{L}^{\otimes (-1)} \cong \mathcal{O} . \]
So \( \widetilde{\mathcal{L}} \cong p_2^* \mathcal{O} \cong \mathcal{O} \). \qed

\begin{remark}
  The same argument shows that whenever \( X,Y \) are irreducible smooth, \( \mathcal{L} \) is a line bundle on \( X \times Y \) such that \( \mathcal{L}|_{X \times \eta} \) is trivial, where \( \eta \in Y \) is the generic point, then \( \mathcal{L} \cong p_2^* \mathcal{N} \) for some \( \mathcal{N} \in \operatorname{Pic}(Y) \).

  There is a better version of this statement, namely the \textit{See-saw Theorem}: Let \( X \) be proper and \( Y \) finite type, reduced, connected.
  Let \( \mathcal{L} \) be a line bundle on \( X \times Y \). Then \( \mathcal{L} \cong p_2^* \mathcal{N} \) if and only if \( \mathcal{L}|_{X \times \left\lbrace s \right\rbrace} \) is trivial for every closed point \( s \) of \( Y \).

    Note that a line bundle on \( X \times Y \) corresponds to a map \( Y \to \operatorname{Pic}(X) \). So the See-saw Theorem is a statement about separatedness of \( \operatorname{Pic}(X) \).
\end{remark}

\subsection{Dual isogenies}
Let \( \phi \colon E \to E' \) be an isogeny. There is an induced map \( \phi^* \colon \operatorname{Pic}^0(E') \to \operatorname{Pic}^0(E) \) with \( \mathcal{L} \mapsto \phi^* \mathcal{L} \).
The Abel-Jacobi map gave an isomorphism \( E( \overline{k} ) \cong \operatorname{Pic}^0(E) \).
So there is a unique \( \hat{\phi} \) fitting into the diagram
% \[
% \begin{tikzcd}
%   \operatorname{Pic}^0 (E') \arrow[r,"\phi^*"] & \operatorname{Pic}^0(E) \\
%   E'( \overline{k} ) \arrow[u,"\operatorname{AJ}","\cong"'] \arrow[r,"\hat{\phi}"] & E( \overline{k} ) \arrow[u,"\operatorname{AJ}","\cong"'] .
% \end{tikzcd}
% \]
\begin{definition}
  The map \( \hat{\phi} \) above is the \textit{dual isogeny} of \( \phi \).
\end{definition}
We should justify the terminology.
Namely, we have described \( \hat{\phi} \) on points, but we should verify that it is an isogeny.
\begin{proposition}
  There exists a unique isogeny \( \hat{\phi} \colon E' \to E \) such that \( \hat{\phi}|_{E(\overline{k})} \) agrees with the above definition.
\end{proposition}
\begin{remark}
  If we had promoted the Abel-Jacobi map to a morphism of \textit{schemes}, this would be completely transparent.
\end{remark}
\textit{Proof.}
The idea is to characterize \( \hat{\phi} \) by
\[ [ \deg \phi ] = \hat{\phi} \circ \phi . \]
Let us check that our definition satisfies this:
Choose \( P \in E(\overline{k}) \), and let \( Q = \phi(P) \).
Then
\begin{align*}
 \hat{\phi}(Q) & = \operatorname{AJ}^{-1} \circ \phi^* \circ \operatorname{AJ}(Q)  = \operatorname{AJ}^{-1} \left( \phi^* \mathcal{O}(Q-O) \right).
\end{align*}
Now, using the fact that fibers of \( \phi \) differ by translations via elements of \( \ker \phi \),
\[ \phi^* \mathcal{O}(Q-O)
  = \mathcal{O}(\phi^{-1} Q - \phi^{-1} O' ) = \mathcal{O} ( \sum n_R (R \oplus P) - \sum n_R R )
  = \mathcal{O}(P-O)^{\otimes (\sum n_R )}
 = \operatorname{AJ}(P)^{\otimes \operatorname{deg} \phi} . \]
Taking \( \operatorname{AJ}^{-1} \), we see that our definition satisfies the condition.

To see that \( \hat{\phi} \) exists, we have to show that \( \hat{\phi} \) exists in the diagram
\[
\begin{tikzcd}
  E \arrow[rr,"\phi"] \arrow[dr,"{[\deg \phi]}"'] & & E' \arrow[dl,dashed,"{\hat{\phi}}"] \\
  & E . & 
\end{tikzcd}
\]
This is equivalent to showing that \( [\deg \phi] \) is invariant under translation by \( \ker \phi \), i. e. that \( [\deg \phi] (\ker \phi) = 0 \).
But this is Lagrange's Theorem!
In the separable case it is literally Lagrange's Theorem.
One could also make sense of Lagrange's Theorem in the inseparable case, but alternatively, if \( \phi = \operatorname{Frob} \) we can directly factorize,
\[
\begin{tikzcd}
  E \arrow[rr,"{\operatorname{Frob}}"] \arrow[dr,"{[p]}"'] & & E^{(p)} \arrow[dl,"{\exists !}",dashed] \\
  & E . & 
\end{tikzcd}
\]
The existence of \( \hat{\phi} \) is stable under composition: \( \widehat{\phi \circ \psi} = \hat{\phi} \circ \hat{\psi} \).
And any \( \phi \) is the composition of an inseparable and Frobenius. \qed

Summary of properties of the dual isogeny:
\begin{enumerate}
\item \( \widehat{\phi \circ \psi} = \hat{\phi} \circ \hat{\psi} \)
\item \( \widehat{\phi + \psi} = \hat{\phi} + \hat{\psi} \).
\item \( \phi \circ \hat{\phi} , \hat{\phi} \circ \phi = [ \deg \phi ] \).
\item \( \deg \hat{\phi} = \deg \phi \)
\end{enumerate}
In particular, \( \operatorname{End}(E) \) has a positive anti-involution \( \hat{\cdot} \).
\begin{remark}
  In higher dimension, given \( \phi \colon A \to B \) there is a dual isogeny \( \operatorname{Pic}^0(B) = \hat{B} \to \hat{A} = \operatorname{Pic}^0(A) \).
  But there are not isomorphisms \( A \cong \hat{A} \) and \( B \cong \hat{B} \) in general; this is a feature of dimension 1.
\end{remark}
\subsection{Example: The Poncelet Porism}
Assume \( \operatorname{char} k \neq 2 \).
Let \( C_1, C_2 \subseteq \mathbb{P}^2 \) be smooth conics (i. e. of degree 2) which are not tangent to each other.
Let \( P_1 \in C_1 \), and consider the following process.
Given \( P_k \), draw the line \( l \) through \( P_k \) that is tangent to \( C_2 \), and let \( P_{k+1} \) be the other intersection point of \( l \) with \( C_1 \).
\todo[inline]{Insert picture}
\begin{proposition}
  Whterh this process is periodic is independent of \( P \).
\end{proposition}
\begin{example}
  Start with an equilateral triangle, let \( P \) be a vertex, \( C_1 \) the circumscribed circle and \( C_2 \) the  inscribed circle.
  Then the process is evidently periodic (with period 3). The same is true if we had started with a regular \( n \)-gon instead of a triangle.
  More interestingly, if we started with say a random quadrilateral that has an incribed and circumscribed circle, this is still true, but not for obvious symmetry reasons.
\end{example}
\begin{example}
  Take two circles centered at the same point. Then the process is not periodic for most ratios of radii.
\end{example}
\textit{Proof.}
Consider \( \check{ \mathbb{P}}^2 = \left\lbrace \text{lines in } \mathbb{P}^2 \right\rbrace \).
The dual conic to \( C_2 \) is
\[ \check{C}_2 = \left\lbrace \text{lines tangent to } C_2 \right\rbrace \subseteq \check{ \mathbb{P}}^2 , \]
which is also a smooth conic.
Now consider
\[
\widetilde{C} = \left\lbrace (P,l) \in C_1 \times \check{C}_2 : P \in l \right\rbrace \subseteq \mathbb{P}^2 \times \check{\mathbb{P}}^2 .
\]
In other words, \( \widetilde{C} \) is the intersection of \( C_1 \times \check{C}_2 \) with the incidence plane \( H = \left\lbrace (P,l) \in \mathbb{P}^2 \times \check{\mathbb{P}}^2 : P \in l \right\rbrace \).
The point now is that
\begin{enumerate}
\item \( \widetilde{C} \) is a smooth projective curve.
\item The genus of \( \widetilde{C} \) is 1.  
\end{enumerate}
To check this, consider the projection map \( \pi \colon \widetilde{C} \to C_1 \). The fiber over a point \( P \in C_1 \) is given by
\[ \pi^{-1} \left\lbrace P \right\rbrace = \left\lbrace (P,l) : l \text{ tangent to } C_2 \text{ and } P \in l \right\rbrace . \]
There are two lines \( l_1,l_2 \) tangent to \( C_2 \) that pass through \( P \).
So \( \pi^{-1} \left\lbrace P \right\rbrace \) has two distinct points, unless \( P \in C_1 \cap C_2 \).
In fact, \( \pi \) is a branched cover of degree 2, ramified at the 4 points where \( C_1 \) and \( C_2 \) intersect.
The Riemann-Hurwitz formula then tells you that the genus of \( \widetilde{C} \) is 1.

We can now describe the above process as follows:
Consider the automorphism
\begin{align*}
  \tau \colon \widetilde{C} & \to \widetilde{C} \\
  (P,l) & \mapsto (P',l)
\end{align*}
that sends \( (P,l) \) to the other intersection of \( C_1 \) with \( l \).
The key point is now that \( \tau \) has no fixed points (because \( C_1 \) is not tangent to \( C_2 \)).
Therefore, \( \tau = t_{M} \) for some \( M \in \widetilde{C}(\overline{k}) \).
Now \( \tau^N = t_{[N]M} \). Either \( \tau^N = \operatorname{id} \) for some \( N \) (or equivalently, \( M \) is torsion), or \( \tau^N \) has no fixed points for all \( N \).
These two situations correspond to whether the process we described is periodic or not. \qed

%%% Local Variables:
%%% mode: latex
%%% TeX-master: "elliptic-curves"
%%% End:
